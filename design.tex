\section{Design Guidelines}
(overview and definitions)

* single-page design

* information-searching / focus+context design

* walk-through first use instead of tutorial (see, copy, do) /cite{?}

* use existing terminology (but also define it)

In the ideal modeling toolkit, trained machine learning models can be tested alongside dynamical systems formulations of behavioral theory. 
Though greater accuracy may often be achieved using a data-driven model, the theoretical formulation may provide additional insight into the system behavior, and connections to conceptual models can be drawn. 
Enabling easy system configuration allows researchers to leverage the best of all fields, creating a balance between predictive accuracy and model clarity depending on the user’s needs. 
The system configuration must have an intuitive high-level representation, and a highly-detailed interface for defining specifics. 
The user should be given the option to make many modeling assumptions to maximize ease of use, or to make few assumptions to ensure modeling specificity. 
Model re-use and extension must be supported by allowing concepts to operate in interchangeable modules, requiring carefully-defined standards for interface between modules. 
The ideal toolkit also supports model evaluation, comparison, optimization, and exploration to aid the user.

\subsection{UI considerations}
Ease of use and intuitive user interface is a primary design consideration for existing softwares for modeling and simulation in systems theory. (reference & cite vensim, etc, from here) 
The ease of use of a system is one of the most important factors in the adoption of the system. (cite technology acceptance model?) 
The trans-disciplinary nature of a human-behavior modeling toolkit may require special consideration in order to be usable by those who may be unfamiliar with modeling and simulation concepts.

TODO: Figure: path2flow.png example of the rise in model complexity for the theory of planned behavior when mathematical assumptions are made explicit. (from: Ajzen 1985,1991,2002 (left), Nandola et al. 2013 (right))
